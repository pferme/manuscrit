%\documentclass[a4paper, 11pt, draft]{memoir}
\documentclass[a4paper, 11pt]{memoir}
\semiisopage

%% Highlight overfull hbox
%\overfullrule=0.5mm
%% Show labels
%\usepackage{showkeys}

\usepackage[utf8x]{inputenc}
\usepackage[french,english]{babel}
%\usepackage[UKenglish]{babel}
\usepackage[T1]{fontenc}
\usepackage{makeidx}
\makeindex

% Customization
\usepackage{lmodern}
\usepackage{libertine}
\usepackage[scaled=.87]{inconsolata}
\chapterstyle{madsen}

\usepackage{subfig}
\usepackage{float}
%\floatstyle{boxed}
\restylefloat{figure}

\pdfobjcompresslevel 0

\let\theoldbibliography\thebibliography
\renewcommand\thebibliography[1]{
  \theoldbibliography{#1}
  \setlength{\parskip}{0pt}
  \setlength{\itemsep}{4pt plus 0.3ex}
  \small
}

\usepackage{xcolor, graphicx}
\usepackage{multirow}
\usepackage[pagebackref]{hyperref}
\renewcommand*{\backref}[1]{}
\renewcommand*{\backrefalt}[4]{\small Citations: \S{}~#2}
\hypersetup{colorlinks=true, linkcolor=black!50!blue, urlcolor=black!50!red, citecolor=black!50!purple, breaklinks=true}
%\hypersetup{colorlinks=true, linkcolor=black, urlcolor=black, citecolor=black, breaklinks=true}
\hypersetup{pdftitle={Approximation Algorithms for Channel Coding and Non-signaling Correlations},
  pdfauthor={Paul Fermé},
  pdfsubject={Approximation Algorithms, Information Theory}}
\usepackage{microtype}
\usepackage{longtable}

% numbering
\setsecnumdepth{subsubsection}
\maxtocdepth   {subsection}
\setlength{\parskip}{5pt}
\usepackage{enumerate}
\makeatletter
  \renewcommand{\counterwithin}{\@ifstar{\@csinstar}{\@csin}}
\makeatother

\usepackage{doi}
\usepackage{amsmath, amssymb}%, mathrsfs, mathtools}
\usepackage{amsthm} % For theorem style
\usepackage{thmtools}
\usepackage{thm-restate}
\usepackage{comment}


% theorems, definitions
\declaretheorem[numberwithin=chapter]{theorem}
\declaretheorem[sibling=theorem]{lemma}
\declaretheorem[sibling=theorem]{corollary}
\declaretheorem[sibling=theorem]{proposition}
\declaretheorem[numberwithin=chapter,style=definition]{definition}
\declaretheorem[sibling=definition]{hypothesis}

\theoremstyle{remark}
\newtheorem*{rk}{Remark}

% References
\usepackage[capitalise]{cleveref}
\usepackage{pdfpages}
\usepackage{xspace}

\input frtoc
\input macros

%%% Personnal packages
% Style CLEAN
\usepackage{bm}
\usepackage{bbm}
% Solving conflict while uploading physics package
\let\trace=\undefined
\usepackage{physics}
\usepackage{enumitem}

% tikz
\usepackage{tikz}
\usepackage{tikzpeople}
\usetikzlibrary{positioning,patterns,shapes,arrows,quotes,angles,calc}
\usetikzlibrary{spy}
\tikzstyle{block} = [draw, fill=white, rectangle, minimum height=2em, minimum width=3em]
\tikzstyle{bigblock} = [draw, fill=white, rectangle, minimum height=3em, minimum width=4em]
\tikzstyle{Bigblock} = [draw, fill=white, rectangle, minimum height=5em, minimum width=6em]
\tikzstyle{input} = [coordinate]
\tikzstyle{output} = [coordinate]
\tikzstyle{pinstyle} = [pin edge={to-,thin,black}]
\usepackage{pgfplots}
\pgfplotsset{compat = newest}

\begin{document}
\pagenumbering{roman}
\setcounter{page}{0}
\includepdf{garde.pdf}
\frontmatter

\setcounter{page}{1}
\newpage
\thispagestyle{empty}
\mbox{}
\newpage
\thispagestyle{empty}
%\cleardoublepage
%%%%%%%%%%%%%
% Décidaces %
%%%%%%%%%%%%%
\vspace*{\stretch{1}}
\begin{flushright}
  %À \ldots
\begin{otherlanguage}{french}
  Dédicace,\\
  peut-être\ldots
\end{otherlanguage}
\end{flushright}
\vspace*{\stretch{2}}
%%%%%%%%%%%%%

\input abstract

\input acknowledgements

%\cleardoublepage
%\frenchtableofcontents
\cleardoublepage
\tableofcontents

\cleardoublepage
\input symbols

%\begin{otherlanguage}{french}
%\chapter{Résumé substantiel en Français}
%\label{chap:resume-fr}
%\input chap-resume
%\end{otherlanguage}

\mainmatter
\pagestyle{ruled}

\chapter{Introduction}
\addcontentsline{tof}{chapter}{\protect\numberline{\thechapter} Introduction}
\label{chap:introduction}

\input chap-introduction

\cleardoublepage
{\let\newpage\relax
\chapter{Background}
\addcontentsline{tof}{chapter}{\protect\numberline{\thepart} Préliminaires}
\label{chap:background}
}

\input chap-background

\cleardoublepage
{\let\newpage\relax
\chapter{Tight Approximation Guarantees for Concave Coverage Problems}
\addcontentsline{tof}{chapter}{\protect\numberline{\thepart} TODO}
\label{chap:MaxCoverage}
}

\input chap-MaxCoverage

\cleardoublepage
{\let\newpage\relax
\chapter{Multiple-access Channel Coding with Non-signaling Correlations}
\addcontentsline{tof}{chapter}{\protect\numberline{\thepart} TODO}
\label{chap:MAC}
}

\input chap-MAC

\cleardoublepage
{\let\newpage\relax
  \chapter{Broadcast Channel Coding with Non-signaling Correlations}
\addcontentsline{tof}{chapter}{\protect\numberline{\thepart} TODO}
\label{chap:Broadcast}
}

\input chap-Broadcast

%%%%%%%%%%%

\chapter*{Conclusion}
\addcontentsline{toc}{chapter}{Conclusion}
\addcontentsline{tof}{chapter}{Conclusion}

\input chap-conclusion

\bibliographystyle{alphaurl}
\bibliography{these}
\addcontentsline{tof}{chapter}{Bibliographie}
%\printindex
%\addcontentsline{tof}{chapter}{Index en anglais}
\backmatter
%\listoffigures
%\addcontentsline{tof}{chapter}{Liste des figures}
%\clearpage
%\listoftables
%\addcontentsline{tof}{chapter}{Liste des tableaux}
\end{document}
% vim: spl=en
