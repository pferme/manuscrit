\begin{otherlanguage}{french}
Avec le nombre croissant d'appareils connectés et l'explosion du volume de données échangées, le besoin de communication efficace et fiable n'a jamais été aussi critique que dans le monde d'aujourd'hui. Le bruit, qui provient d'imperfections physiques, d'interférences entre les signaux ou même de \emph{lépidoptères}~\cite{Hopper81}, est l'un des principaux obstacles à surmonter.

La théorie de l'information établie par Shannon dans son ouvrage fondateur~\cite{Shannon48} apporte des réponses claires et précises à la quantité de données pouvant être transmises par des canaux point à point bruités. Le taux asymptotique auquel les informations peuvent être envoyées par de multiples copies indépendantes d'un canal est entièrement caractérisé par une quantité mathématique appelée \emph{capacité}. Bien que l'on ne puisse espérer dépasser cette limite fondamentale, le développement de codes permettant d'atteindre la capacité du canal n'est pas une tâche simple dans la pratique, car la construction du code par Shannon n'était que probabiliste et le décodage est inefficace en fonction du nombre de copies du canal.

Néanmoins, en limitant l'étude aux canaux symétriques binaires, la recherche sur les codes correcteurs d'erreurs a conduit à des solutions approchant la capacité avec des procédures de codage et de décodage efficaces, telles que les codes de contrôle de parité à faible densité~\cite{Gallager62}, les codes turbo~\cite{BG96} ou, plus récemment, les codes polaires~\cite{Arikan09}.

Une autre approche, développée par Barman et Fawzi dans~\cite{BF18}, se concentre sur une copie unique d'un canal arbitraire. L'objectif consiste alors à maximiser la probabilité de transmettre avec succès un message par ce canal pour tous les codeurs et décodeurs possibles, ce que l'on appelle le problème du codage de canal. Cette tâche algorithmique étant \textrm{NP}-difficile, ils ont proposé un algorithme d'approximation en temps polynomial atteignant un ratio de $1-e^{-1}$, qui ne peut être augmenté si $\textrm{P}\not=\textrm{NP}$ grâce à un lien avec le problème de la couverture maximale~\cite{Feige98}.

Suite à ces travaux, le problème du décodage de listes de paramètre $\ell$~\cite{Elias57,Wozencraft58}, où l'on demande au décodeur de produire une liste de $\ell$ suppositions possibles au lieu de donner le bon message d'entrée, a été étudié d'un point de vue algorithmique dans~\cite{BFGG20}. De même, ils ont développé un algorithme d'approximation de ratio $1-\frac{\ell^{\ell}e^{-\ell}}{\ell!}$ fonctionnant en temps polynomial, et ont prouvé que ce ratio ne peut pas être augmenté sous l'hypothèse de la conjecture des jeux uniques~\cite{Khot02}. Plus généralement, une classe plus large de problèmes d'optimisation combinatoire, appelée couverture multiple maximale de paramètre $\ell$, relève de l'analyse de~\cite{BFGG20}.

Une extension naturelle du problème de couverture multiple maximale de paramètre $\ell$ consiste à maximiser la quantité $\sum_{a \in [n]} w_a\varphi(\abs{\set{i \in S : a \in T_i}})$ sur les sous-ensembles $S \subseteq [m]$ de cardinal $k$, pour $\varphi$ une fonction croissanta concave quelconque ; le problème de couverture multiple maximale de paramètre $\ell$ est retrouvé en prenant $\varphi(x) := \min(x,\ell)$. En termes d'interprétation du codage de canal, cela correspond à ce que nous appelons le décodage de liste de paramètre $\varphi$, où la restriction de longueur sur la liste de suppositions du décodeur dans le décodage de liste de paramètre $\ell$ est remplacée par une probabilité $\frac{\varphi(\ell)}{\ell}$ de décodage correct d'une liste de suppositions de taille variable $\ell$. Ce problème sera le premier sujet d'étude de la thèse.

Au début du vingtième siècle, la mécanique quantique a ébranlé les fondements physiques du monde et révolutionné la compréhension de la nature à l'échelle atomique et subatomique. L'une de ses caractéristiques les plus fascinantes est la notion d'\emph{intrication}. Deux particules intriquées ont la particularité d'être corrélées, même lorsqu'elles sont éloignées l'une de l'autre, d'une manière qui ne peut être expliquée par les lois de la physique non quantique.

Einstein, Podolsky et Rosen~\cite{EPR35} ont été les premiers à découvrir ce phénomène, qui impliquait soit que la description de la réalité physique par la mécanique quantique n'était pas complète, soit que deux grandeurs physiques incompatibles n'avaient pas simultanément une valeur concrète. De manière surprenante et contrairement à l'objectif initial de ce qui est devenu le paradoxe d'Einstein-Podolsky-Rosen~\cite{EPR35}, c'est ce dernier qui décrit correctement la réalité physique atomique et subatomique.

Bell a montré dans~\cite{Bell64} qu'aucune théorie des variables cachées, où l'on pourrait associer des variables locales inconnues à chacune des particules intriquées, ne suffirait à expliquer leur corrélation lorsqu'elles sont éloignées l'une de l'autre. Cette nature non locale de la physique quantique est en contradiction avec le principe de localité de la physique classique, qui stipule que tout objet ne peut être influencé que par son environnement immédiat. Les résultats de~\cite{Bell64} ont été généralisés dans ce que l'on appelle les inégalités CHSH~\cite{CHSH69}, qui, si elles sont violées, impliquent qu'une théorie des variables cachées ne peut pas expliquer les corrélations entre les particules. Les violations de ces inégalités ont été pratiquement réalisées par Aspect et al.~\cite{ADG82}, ce qui a finalement prouvé la nature non locale de la physique quantique.

Une autre particularité de la mécanique quantique est que toute mesure des propriétés physiques d'un système quantique le perturbe. En outre, dans le cas de particules intriquées, la mesure de l'une des particules affectera également l'autre. Cette \emph{action étrange à distance}, comme l'a décrite Einstein, ne viole pas la relativité restreinte, car il est impossible de transmettre des informations par ce biais. Le résultat d'une mesure quantique étant intrinsèquement aléatoire, seule une corrélation est obtenue entre les particules.

Cependant, l'intrication peut être utilisée pour améliorer la communication. Un qubit, l'unité quantique de base de l'information, ne peut à lui seul transmettre plus d'un bit d'information~\cite{Holevo73}. Cependant, Bennett et Wiesner ont montré dans~\cite{BW92} que si l'intrication quantique est partagée entre les parties, sous la forme de ce que l'on appelle une paire EPR, il est possible de transmettre deux bits d'information en utilisant un seul qubit. C'est ce que l'on appelle aujourd'hui le codage superdense. Pour les canaux point à point (classiques), l'intrication quantique partagée entre l'émetteur et le récepteur peut augmenter la probabilité de succès optimale du codage de canal \cite{CLMW10,PLMKR11}, bien qu'elle n'augmente pas la capacité du canal~\cite{BBCJPW93,BSST99}.

On peut également abstraire l'intrication quantique en corrélations \emph{non-signalantes}~\cite{Tsirelson80,PR94}. Afin de comprendre ces corrélations, il est utile d'examiner l'interprétation de l'inégalité CHSH mentionnée précédemment. Nous considérons un jeu à deux joueurs avec Alice et Bob. Un arbitre donne un bit uniformément aléatoire $x$ (resp. $y$) à Alice (resp. Bob), et leur but commun est de choisir respectivement les bits $a$ et $b$ tels que $a \oplus b = x \land y$ sans communiquer, où $\oplus$ dénote la disjonction exclusive. Il est facile de voir qu'aucune stratégie ne peut conduire à une meilleure probabilité de succès que $\frac{3}{4}$, même en supposant que des variables cachées sont partagées entre les joueurs. Cependant, comme le montre~\cite{CHSH69}, si Alice et Bob partagent une paire EPR intriquée, ils peuvent appliquer des mesures intelligentes et atteindre une probabilité de succès de $\cos^2\left(\frac{\pi}{8}\right) \simeq 0.85$, ce qui en fait ne peut pas être surpassé~\cite{Tsirelson80}. Notons que la corrélation $P(ab|xy)$ décrivant le résultat de cette stratégie quantique confirme qu'aucune information n'est transmise à l'aide de l'intrication quantique. En effet, on peut montrer que la marginale d'un joueur est indépendante de l'entrée de l'autre joueur, c'est-à-dire que $\sum_bP(ab|xy)=\sum_bP(ab|xy')$ et $\sum_aP(ab|xy)=\sum_aP(ab|x'y)$ pour tous $a,b,x,y,x',y'$. Les corrélations non-signalantes sont définies comme l'ensemble des distributions conjointes $P(ab|xy)$ qui satisfont les égalités précédentes. Elles incluent naturellement les stratégies quantiques, mais sont en fait plus fortes, car on peut même obtenir une probabilité de succès de $1$ pour le jeu CHSH en utilisant une stratégie générale non-signalante, simplement en définissant $P(ab|xy):=\frac{1}{2}$ si $a \oplus b = x \land y$, et $P(ab|xy):=0$ sinon~\cite{PR94}.

Les corrélations non-signalantes générales ne représentent pas la réalité physique, car elles impliquent des comportements non locaux plus forts que ceux qui se produisent dans le cadre de la mécanique quantique. Cependant, elles présentent un grand intérêt théorique. Notamment, la description de l'ensemble des corrélations non-signalantes est beaucoup plus simple que celle des corrélations quantiques, car elles sont caractérisées par de simples contraintes linéaires ; voir~\cite{BCPSW14} pour une revue générale sur la non-localité.

L'algorithme d'approximation pour le problème du codage de canal dans~\cite{BF18} s'appuie en fait sur des codes non-signalants. Comme la recherche du meilleur code non-signalant pour un canal est un programme linéaire, il peut être résolu exactement en temps polynomial. Ils ont mis au point une stratégie pour transformer ce code non-signalant en un code classique, en perdant au maximum un facteur $1-e^{-1}$ dans la probabilité de succès. Un énoncé plus précis de cette stratégie implique en fait que les zones de capacité avec ou sans assistance non-signalante sont les mêmes, retrouvant un résultat de~\cite{Matthews12}. Ce lien inattendu entre les algorithmes d'approximation pour le problème du codage de canal et les corrélations non-signalantes est la motivation principale du reste de la thèse. En particulier, nous développerons ces analyses dans les principaux scénarios de communication en réseau.

La théorie de l'information des réseaux, qui vise à comprendre la communication sur des canaux à émetteurs multiples et à récepteurs multiples, a été étudiée pour la première fois par Shannon dans~\cite{Shannon61}, dans le cas particulier des canaux deux-à-deux (connus aujourd'hui sous le nom de canaux d'interférence). Plus tard, Cover~\cite{Cover72} a introduit les canaux de diffusion, avec plusieurs récepteurs mais un seul émetteur. Ahlswede~\cite{Ahlswede73} et indépendamment Liao~\cite{Liao73} ont étudié le scénario inverse des canaux à accès multiple, avec plusieurs émetteurs mais un seul récepteur. Ces canaux plus complexes permettent une meilleure modélisation des communications interconnectées réelles.

Contrairement à la situation point-à-point, la non-localité peut augmenter la capacité des canaux en réseau. Quek et Shor ont montré dans~\cite{QS17} l'existence de canaux d'interférence à deux émetteurs et deux récepteurs avec des écarts entre leurs zones de capacité classique, avec assistance d'intrication quantique et avec assistance non-signalante. À la suite de ce résultat, Leditzky et al. ont montré que l'intrication quantique partagée entre les deux émetteurs d'un canal à accès multiple peut strictement agrandir sa zone de capacité. D'autres exemples de canaux en réseau pour lesquels l'intrication augmente leur zone de capacité ont été étudiés dans~\cite{Noetzel20,ND20}.

Nous aborderons le problème du codage pour les canaux à accès multiple et les canaux de diffusion. Nous étudierons l'impact des corrélations non-signalantes sur leurs capacités, ainsi que leurs liens avec les algorithmes d'approximation des problèmes de codage associés.

\paragraph{Contributions} Dans le Chapitre \ref{chap:MaxCoverage}, nous considérons la généralisation suivante du problème de couverture multiple maximale de paramètre $\ell$ dépendant d'une fonction croissante concave $\varphi$ : étant donné des sous-ensembles $T_1, \ldots, T_m$ d'un univers $[n]$, des poids strictement positifs $w_a$ sur l'univers $[n]$ et d'un entier $k$, l'objectif est de trouver un sous-ensemble $S \subseteq [m]$ de taille $k$ qui maximise $C^{\varphi}(S) \coloneqq \sum_{a \in [n]}w_a\varphi(\abs{S}_a)$, où $\abs{S}_a : = \abs{\set{i \in S : a \in T_i}}$ ; le problème de couverture multiple maximale de paramètre $\ell$ est retrouvé en prenant $\varphi(x) := \min(x,\ell)$. Pour tout $\varphi$ de ce type, nous fournissons un algorithme efficace qui permet d'obtenir un rapport d'approximation égal au \emph{rapport de concavité de Poisson} de $\varphi$, défini par $\alpha_{\varphi} := \min_{x \ dans \mathbb{N}^*} \frac{\mathbb{E}[\varphi(\Poi(x))]}{\varphi(\mathbb{E}[\Poi(x)])}$. En complément de cette garantie d'approximation, nous établissons un résultat de \textrm{NP}-difficulté lorsque $\varphi$ croît de manière sous-linéaire. Appliqué au codage de canal, et plus spécifiquement au décodage de liste de paramètre $\varphi$, où la restriction de longueur sur la liste de suppositions du décodeur dans le décodage de liste de paramètre $\ell$ est remplacée par une probabilité $\frac{\varphi(\ell)}{\ell}$ de décoder correctement une liste de suppositions de taille variable $\ell$, nous obtenons une garantie stricte d'approximation $\alpha_{\varphi}$ pour la classe de canaux $W$ de la forme $W(y|x) = \frac{1}{t}$ pour $y \in T_x$ avec $\abs{T_x}=t$ et $W(y|x) = 0$ ailleurs. Notre résultat dépasse ce cadre particulier et nous l'illustrons par des applications aux problèmes d'allocation de ressources distribuées, aux problèmes de maximisation de la prospérité et au vote basé sur l'approbation pour des règles générales.

Dans le Chapitre \ref{chap:MAC}, nous abordons le problème du codage pour les canaux à accès multiple. Nous montrons tout d'abord qu'il ne peut pas être approximé en temps polynomial  avec un ratio constant, sous certaines hypothèses de complexité sur les clauses aléatoires. Ensuite, nous étudions l'influence des corrélations non signalantes entre les parties. Nous développons un programme linéaire calculant la probabilité de succès optimale pour le codage sur $n$ copies d'un canal d'accès multiple $W$ dont la taille croît polynomialement en $n$. La résolution de ce programme linéaire nous permet d'obtenir des bornes inférieures pour les canaux à accès multiple. En appliquant cette méthode au canal additionneur binaire, nous montrons qu'en utilisant une assistance non-signalante, la somme des taux $frac{\log_2(72)}{4} \simeq 1,5425$ peut être atteinte même avec une erreur nulle, ce qui dépasse la capacité maximale de la somme des taux de $1,5$ dans le cas sans assistance. Pour les canaux bruités, où la région de capacité avec assistance non-signalante et sans erreur est triviale, nous pouvons utiliser des codes concaténés pour obtenir des points réalisables dans la zone de capacité. Appliqués à une version bruitée du canal additionneur binaire, nous montrons que l'assistance non-signalante améliore encore la capacité. En complément de ces résultats de faisabilité, nous donnons une borne supérieure à la zone de capacité avec assistance non-signalante qui a la même expression que la région sans assistance, sauf que les entrées du canal ne sont pas obligées d'être indépendantes. Enfin, nous montrons que la zone de capacité avec assistance non-signalante partagée uniquement entre chaque émetteur et le récepteur indépendamment est la même que sans assistance.

Dans le Chapitre \ref{chap:Broadcast}, nous abordons le problème du codage pour les canaux de diffusion. Nous montrons tout d'abord que lorsque l'assistance non-signalante est partagée uniquement entre les décodeurs, la zone de capacité ne change pas. Pour la classe des canaux de diffusion déterministes, nous décrivons un algorithme d'approximation de ratio $(1-e^{-1})^2$ fonctionnant en temps polynomial, et nous montrons que la zone de capacité pour cette classe est la même avec ou sans assistance non-signalante. Enfin, nous montrons que dans le modèle d'accès par valeur, nous ne pouvons pas obtenir un meilleur ratio d'approximation que $\Omega\left(\frac{1}{\sqrt{m}}\right)$ en temps polynomial pour le problème général du codage des canaux de diffusion, avec $m$ la taille de l'une des sorties du canal.
\end{otherlanguage}{french}
