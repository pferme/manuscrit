In this thesis, we have studied several channel coding problems, with and without non-signaling assistance, from the points of view of algorithmic approximation and of capacity regions.

In Chapter \ref{chap:MaxCoverage}, we have introduced a generalization of the maximum $\ell$-multi-coverage problem which entails maximizing the quantity $\sum_{a \in [n]}  w_a\varphi(\abs{\set{i \in S : a \in T_i}})$ over subsets $S \subseteq [m]$ of cardinality $k$. We have shown that when $\varphi$ is normalized, nondecreasing and concave, we can obtain an approximation guarantee given by the \emph{Poisson concavity ratio} $\alpha_{\varphi} := \min_{x \in \mathbb{N}^*} \frac{\mathbb{E}[\varphi(\Poi(x))]}{\varphi(\mathbb{E}[\Poi(x)])}$ and we showed it is tight for sublinear functions $\varphi$ if $\textrm{P}\not=\textrm{NP}$. Applied to channel coding, and more specifically to $\varphi$-list-decoding, where the length restriction on the list of guesses of the decoder in $\ell$-list-decoding is replaced by a probability $\frac{\varphi(\ell)}{\ell}$ of correctly decoding a list of guesses of variable size $\ell$, we have obtained a tight approximation guarantee $\alpha_{\varphi}$ for the class of channels $W$ of the form $W(y|x) = \frac{1}{t}$ for $y \in T_x$ with $\abs{T_x}=t$ and $W(y|x) = 0$ elsewhere. A natural open question is whether the \textrm{NP}-hardness guarantee can be extended for $\varphi(n) \not= o(n)$. Another open problem is to extend the $\alpha_{\varphi}$-approximation algorithm for $\varphi$-list-decoding on all channels.

In Chapter \ref{chap:MAC}, we have shown that the multiple-access channel coding problem cannot be approximated within any constant ratio under some complexity hypothesis on random clauses. We have shown that optimal non-signaling codes for multiple-access channels can be found in polynomial time in the number of copies of the channel. Applied to the binary adder channel, a non-signaling advantage on its capacity region has been established. We have provided a general single-letter outer bound on the non-signaling capacity region. When non-signaling assistance is not shared between encoders, we have shown that the capacity region is not changed. A remaining open question is whether quantum entanglement may increase the capacity of the binary adder channel. Also, establishing a single-letter formula for the non-signaling assisted capacity of multiple-access channels is the main open question left here, which could be obtained by achieving the provided single-letter outer bound on the non-signaling capacity region.

In Chapter \ref{chap:Broadcast}, we have shown that non-signaling assistance shared only between decoders does not not change the capacity region of broadcast channels. Restricting to deterministic channels, we have provided a $(1-e^{-1})^2$-approximation algorithm for the unassisted coding problem, and we have shown that their capacity region is not changed with non-signaling assistance. In the value query model, we have shown that we cannot achieve a better approximation ratio than $\Omega\left(\frac{1}{\sqrt{m}}\right)$ for the general broadcast channel coding problem, with $m$ an output size of the channel. Our results suggest that non-signaling assistance could improve the capacity region of general broadcast channels, which is left as a major open question. An intermediate result would be to show that it is \textrm{NP}-hard to approximate the broadcast channel coding problem within any constant ratio, strengthening our hardness result without relying on the value query model.

Throughout this thesis, we have shown that links between approximation algorithms for channel coding and non-signaling assisted capacity regions are fruitful in both directions. On the one hand, the existence of approximation algorithms with constant ratio for channel coding results often in the absence of non-signaling advantage on the related capacity regions. On the other hand, the hardness of approximation within any constant ratio for the channel coding problem results often in the existence of a non-signaling advantage on the related capacity regions. Although not a formal equivalence, this strong link between those apparently unrelated domains may be the key to address unsolved problems from both topics. In particular, the step achieved in the understanding of the influence of non-signaling correlations on broadcast channels, solved now for deterministic channels, goes in that direction, and could be extended to other classes of channels. We believe that this connection deserves more attention, for any channel coding problem and even more generally any multi-player games such as the CHSH game, as the radically different approaches from both domains will most likely be profitable for more in-depth comprehension of their respective open questions.
