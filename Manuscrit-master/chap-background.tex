P NP classes?
Hardness reductions?
General probability theory?
General Intro on Information Theory? With definitions of entropies etc. 
Graph notations on partitions?
Groups and orbits and actions (basic stuff)?

Put typical set properties for 1 and 2 parties

Put common lemmas and general stuff: convex order (3.3), negatively associated random variables: recall important properties etc. (clean 3.5.4 with that and Prop 5.15 moved to intro)

Put some basic properties of channels for MAC here

Hypothesis Testing on probas

Put back results from~\cite{BF18} in the background part

General capacity region $\mathcal{C}^{\mathrm{S}}(W)$ from of success probability $\mathrm{S}(W,k_1,k_2)$ and zero error capacity region $\mathcal{C}_0^{\mathrm{S}}(W)$ (easy to define and clean way to generalize): TODO CHANGE IN THE MAIN DOCUMENTS BY APPLI OF THAT GENERAL DEFINITION.

\newpage
\section{Information Theory}
\subsection{Channels}
Formally, a channel $W$ is a conditional probability distribution depending on $n$ inputs belonging to $\mathcal{X}^n$ and $m$ outputs belonging to $\mathcal{Y}^m$, so $W := \left(W(y^m|x^n)\right)_{x^n \in \mathcal{X}^n, y^m \in \mathcal{Y}^m}$ with:
  \[ \forall x^n,y^m, W(y^m|x^n) \geq 0 \text{ and } \forall x^n, \sum_{y^m \in \mathcal{Y}^m} W(y^m|x^n) = 1 \ . \]
We will denote such a channel by $W : \mathcal{X}^n \rightarrow \mathcal{Y}^m$. The tensor product of two channels $W: \mathcal{X}^n \rightarrow \mathcal{Y}^m$ and $W': \mathcal{X}'^n \rightarrow \mathcal{Y}^m$ is denoted by $W \otimes W' : (\mathcal{X}^n \times \mathcal{X}^{\prime n}) \to  (\mathcal{Y}^n \times \mathcal{Y}^{\prime n})$ and defined by:
\[ (W \otimes W')(y^my^{\prime m}|x^nx^{\prime n}) := W(y^m|x^n) \cdot W'^{\prime m}|x^{\prime n}) \ .\]
We denote by $W^{\otimes n}$ the $n$th tensor product of $W$, i.e. $W^{\otimes n} = W \otimes \ldots \otimes W$ with $n$ occurences of $W$.

When $n=m=1$, we will speak of regular or one-way channels. When $n>1,m=1$, we will speak of multiple-access channels. When $n=1,m>1$, we will speak of broadcast channels. If both $n,m$ are greater than $1$, we will speak of interference channels. More specifically, in this thesis, we will focus on the cases of $n=2,m=1$ and $n=1,m=2$, which are at the core of the specificity of network channels.

\subsection{Capacity Regions}
\begin{definition}[Capacity Region $\mathcal{C}^{\mathrm{S}}(W)$ for a success probability $\mathrm{S}(W,k_1,k_2)$]
  \label{defi:generalCapacityRegion}
  A rate pair $(R_1,R_2)$ is $\mathrm{S}$-achievable (for the channel $W$) if:
  \[ \underset{n \rightarrow +\infty}{\lim} \mathrm{S}(W^{\otimes n},\ceil{2^{R_1n}},\ceil{2^{R_2n}}) = 1 \ . \]
  We define the $\mathrm{S}$-capacity region $\mathcal{C}^{\mathrm{S}}(W)$ as the closure of the set of all achievable rate pairs (for the channel $W$).
\end{definition}

\begin{definition}[Zero-Error Capacity Region $\mathcal{C}_0^{\mathrm{S}}(W)$ for a success probability $\mathrm{S}(W,k_1,k_2)$]
  A rate pair $(R_1,R_2)$ is $\mathrm{S}$-achievable with zero-error (for the channel $W$) if:
  \[ \exists n_0  \in \mathbb{N}^*, \forall n \geq n_0, \mathrm{S}(W^{\otimes n},\ceil{2^{R_1n}},\ceil{2^{R_2n}}) = 1 \ . \]
  We define the zero-error $\mathrm{S}$-capacity region $\mathcal{C}^{\mathrm{S}}(W)$ as the closure of the set of all achievable rate pairs with zero-error (for the channel $W$).
\end{definition}

\begin{proposition}[Time-sharing]
  \label{prop:timesharing}
  If the success probability $\mathrm{S}$ satisfies:
  \[ \forall W,W', \forall k_1,k_2,k_1',k_2', \mathrm{S}(W \otimes W',k_1k_1',k_2k_2') \geq \mathrm{S}(W,k_1,k_2) \cdot \mathrm{S}(W',k_1',k_2') \ , \]
  then for all channels $W$, $\mathcal{C}^{\mathrm{S}}(W)$ and $\mathcal{C}_0^{\mathrm{S}}(W)$ are convex.
\end{proposition}
\begin{proof}
  Let $(R_1,R_2)$ and $(R_1',R_2')$, two pairs of $\mathrm{S}$-achievable rational rates for $W$, i.e.:
    \[ \mathrm{S}(W^{\otimes n},\ceil{2^{R_1n}},\ceil{2^{R_2n}}) \underset{n \rightarrow +\infty}{\rightarrow} 1 \text{ and }\mathrm{S}(W^{\otimes n},\ceil{2^{R_1'n}},\ceil{2^{R_2'n}}) \underset{n \rightarrow +\infty}{\rightarrow} 1 \ . \]
    Let $\lambda \in (0,1)$ rational and define $R_{\lambda,i} := \lambda \cdot R_i + (1-\lambda) \cdot R'_i$, let us show that $(R_{\lambda,1},R_{\lambda,2})$ is achievable with non-signaling assistance. Let us call respectively $k_i:=2^{R_i}, k_i' := 2^{R_i'}, k_{\lambda,i} := 2^{R_{\lambda,i}} = k_i^{\lambda}\cdot k_i^{(1-\lambda)}$.

    We have $R_{\lambda,i}n = \lambda \cdot R_in + (1-\lambda) \cdot R_i'n = (\lambda n) \cdot R_i + (1-\lambda)n \cdot R_i'$. This is the idea of \emph{time-sharing}: for $\lambda n$ copies of the channel, we use the strategy with rate $(R_1,R_2)$ and for the $(1-\lambda)n$ other copies of the channel, we use the strategy with rate $(R_1',R_2')$. There exists some $n$ such that $\lambda n,(1-\lambda)n,\lambda n R_i,(1-\lambda)n R_i'$ are integers, since everything is rational. This implies that $k_i^{\lambda n},k_i^{\prime (1-\lambda)n},k_{\lambda,i}^n$ are integers. Thus, by hypothesis, we have that $\mathrm{S}(W^{\otimes n},k^n_{\lambda,1},k^n_{\lambda, 2})$ is larger than or equal to:
\[ \mathrm{S}(W^{\otimes (\lambda n)}, k_1^{\lambda n}, k_2^{\lambda n}) \cdot \mathrm{S}(W^{\otimes ((1-\lambda) n)}, k_1^{\prime (1-\lambda) n}, k_2^{\prime (1-\lambda) n}) \underset{n \rightarrow +\infty}{\rightarrow} 1 \cdot 1 = 1  \ .\]

Thus, since $\mathrm{S}(W^{\otimes n},k^n_{\lambda,1},k^n_{\lambda,2}) \leq 1$, we get the result $\mathrm{S}(W^{\otimes n},k^n_{\lambda,1},k^n_{\lambda,2}) \underset{n \rightarrow +\infty}{\rightarrow} 1$, so $(R_{\lambda,1},R_{\lambda,2})$ is $\mathrm{S}$-achievable for the channel $W$. Finally, since $\mathcal{C}^{\mathrm{S}}(W)$ is defined as the closure of $\mathrm{S}$-achievable rates for the channel $W$, we get that $\mathcal{C}^{\mathrm{S}}(W)$ is convex.

For zero-error capacity regions, since by hypothesis there exists ranks $n_0,n_0'$ such that $\mathrm{S}(W^{\otimes n},\ceil{2^{R_1n}},\ceil{2^{R_2n}}) = 1$ for $n \geq n_0$ and $\mathrm{S}(W^{\otimes n},\ceil{2^{R_1'n}},\ceil{2^{R_2'n}}) = 1$ for $n \geq n_0'$, then in particular we get that for $\lambda n \geq n_0$ and $(1-\lambda) n \geq n_0'$ that $\mathrm{S}(W^{\otimes n},k^n_{\lambda,1},k^n_{\lambda, 2})$ is larger than or equal to:
\[ \mathrm{S}(W^{\otimes (\lambda n)}, k_1^{\lambda n}, k_2^{\lambda n}) \cdot \mathrm{S}(W^{\otimes ((1-\lambda) n)}, k_1^{\prime (1-\lambda) n}, k_2^{\prime (1-\lambda) n}) = 1 \ .\]
It is in particular true for $n \geq n_0 := \max\left(\ceil{\frac{n_0'}{1-\lambda}},\ceil{\frac{n_0}{\lambda}}\right)$. As $\mathrm{S}^{\mathrm{NS}}(W^{\otimes n},k^n_{\lambda,1},k^n_{\lambda,2}) \leq 1$, we have  for all $n \geq n_0$ that $\mathrm{S}^{\mathrm{NS}}(W^{\otimes n},k^n_{\lambda,1},k^n_{\lambda,2}) = 1$, i.e. $(R_{\lambda,1},R_{\lambda,2})$ is $\mathrm{S}$-achievable with zero-error for the channel $W$.  Finally, since $\mathcal{C}_0^{\mathrm{S}}(W)$ is defined as the closure of $\mathrm{S}$-achievable rates with zero-error for the channel $W$, we get that $\mathcal{C}_0^{\mathrm{S}}(W)$ is convex.
\end{proof}

\subsection{Non-signaling probability distributions}
\begin{definition}
  \label{defi:nonsignaling}
  We say that a conditional probability distribution $P(a^n|x^n)$ defined on $\bigtimes_{i=1}^n\mathcal{A}_i \times \bigtimes_{i=1}^n \mathcal{X}_i$ is \emph{non-signaling} if for all $a^n, x^n, \hat{x}^n$, we have
    \[ \forall i \in [n], \sum_{\hat{a}_i}P(a_1\ldots \hat{a}_i \ldots a_n|x_1\ldots x_i \ldots x_n) = \sum_{\hat{a}_i}P(a_1\ldots \hat{a}_i \ldots a_n|x_1\ldots \hat{x}_i \ldots x_n) \ .\]
\end{definition}

\begin{definition}
  Let $P(a^n|x^n)$ a conditional probability distribution defined on $\bigtimes_{i=1}^n\mathcal{A}_i \times \bigtimes_{i=1}^n \mathcal{X}_i$ and $P'(a'^n|x'^n)$ defined on $\bigtimes_{i=1}^n\mathcal{A}'_i \times \bigtimes_{i=1}^n \mathcal{X}'_i$. We define $P \otimes P'$ the tensor product conditional probability distribution defined on $\bigtimes_{i=1}^n(\mathcal{A}_i \times \mathcal{A}'_i) \times \bigtimes_{i=1}^n (\mathcal{X}_i \times \mathcal{X}'_i)$ by $\left(P \otimes P'\right)(a_1a'_1\ldots a_na'_n|x_1x'_1\ldots x_nx'_n) := P(a^n|x^n) \cdot P'(a'^n|x'^n)$.
\end{definition}

\begin{lemma}
  \label{lem:NStensor}
  If both $P$ and $P'$ are non-signaling, then $P \otimes P'$ is non-signaling.
\end{lemma}
\begin{proof}
  Let $a^n \in \bigtimes_{j=1}^n\mathcal{A}_j$, $a'^n \in \bigtimes_{j=1}^n\mathcal{A}'_j$, $x^n \in \bigtimes_{j=1}^n \mathcal{X}_j$, $x'^n \in \bigtimes_{j=1}^n \mathcal{X}'_j$ and $\hat{x}_i \in \mathcal{X}_i$, $\hat{x}'_i \in \mathcal{X}'_i$. Using the fact that $P,P'$ are non-signaling, we have:
  \begin{equation}
    \begin{aligned}
      &\sum_{\hat{a}_i\hat{a}_i'}P(a_1a'_1\ldots \hat{a}_i\hat{a}_i' \ldots a_na'_n|x_1x'_1\ldots x_ix'_i \ldots x_nx'_n)\\
      = &\sum_{\hat{a}_i\hat{a}_i'}  P(a_1\ldots \hat{a}_i \ldots a_n|x_1\ldots x_i \ldots x_n) \cdot P'(a'_1\ldots \hat{a}'_i \ldots a'_n|x'_1\ldots x'_i \ldots x'_n)\\
      = &\left(\sum_{\hat{a}_i}  P(a_1\ldots \hat{a}_i \ldots a_n|x_1\ldots x_i \ldots x_n)\right) \cdot \left(\sum_{\hat{a}'_i}  P'(a'_1\ldots \hat{a}'_i \ldots a'_n|x'_1\ldots x'_i \ldots x'_n)\right)\\
      = &\left(\sum_{\hat{a}_i}  P(a_1\ldots \hat{a}_i \ldots a_n|x_1\ldots \hat{x}_i \ldots x_n)\right) \cdot \left(\sum_{\hat{a}'_i}  P'(a'_1\ldots \hat{a}'_i \ldots a'_n|x'_1\ldots \hat{x}'_i \ldots x'_n)\right)\\
      = &\sum_{\hat{a}_i\hat{a}_i'}\left(P \otimes P'\right)(a_1a'_1\ldots \hat{a}_i\hat{a}'_i \ldots a_na'_n|x_1x'_1\ldots \hat{x}_i\hat{x}'_i \ldots x_nx'_n) \ ,
    \end{aligned}
  \end{equation}
  so $P \otimes P'$ is non-signaling.
\end{proof}
