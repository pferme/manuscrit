% vim: spl=fr
\chapter*{Résumé}
\addcontentsline{toc}{chapter}{Résumé}
\addcontentsline{tof}{chapter}{Résumé}

\begin{otherlanguage}{french}
  Dans cette thèse, nous étudions...
\end{otherlanguage}
\clearpage

\flushright
\bigskip
\hrule \bigskip \bigskip
{\Huge \textbf{\textsf{Abstract}}}
\flushleftright
\addcontentsline{toc}{chapter}{Abstract}
\addcontentsline{tof}{chapter}{Résumé en anglais}
\vspace{1.5cm}

Finding efficient codes is crucial for achieving reliable communication over noisy channels. While error-correcting codes for i.i.d. channels are well understood, the task of approximating the best code for a general channel, i.e. that maximizes the success probability of the encoding and decoding process, is much less developed. This thesis addresses this problem in several scenarios.

For point-to-point channels, optimal approximation algorithms for coding and list-decoding are known. We study a generalization of the latter with a variable list size, compensated by an increasing probability of failure. More generally, we study the coverage problem which entails maximizing $\sum_{a \in [n]}  \varphi(\abs{\set{i \in S : a \in T_i}})$ over subsets $S \subseteq [m]$ of cardinality $k$, for a general nonnegative concave function $\varphi$. We provide an approximation algorithm for that problem of ratio $\alpha_{\varphi} := \min_{x \in \mathbb{N}^*} \frac{\mathbb{E}[\varphi(\Poi(x))]}{\varphi(\mathbb{E}[\Poi(x)])}$, which cannot be improved unless $\textrm{P}\not=\textrm{NP}$.

Generalizing and abstracting quantum entanglement, non-signaling correlations can be used to enhance communication. For multiple-access channels, we show that optimal non-signaling codes can be found in polynomial time in the number of copies of the channel. Applied to the binary adder channel, a non-signaling advantage on its capacity region is established. We provide a general single-letter outer bound on the non-signaling capacity region. When non-signaling assistance is not shared between encoders, we show that the capacity region is not changed.

For broadcast channels, we show that non-signaling assistance shared only between decoders does not not change the capacity region. Restricting to deterministic channels, we provide a $(1-e^{-1})^2$-approximation algorithm, and we show that their capacity region is not changed with non-signaling assistance. In the value query model, we show that we cannot achieve a better approximation ratio than $\Omega\left(\frac{1}{\sqrt{m}}\right)$ for the general broadcast channel coding problem, with $m$ an output size of the channel.

% Surprisingly, the existence of efficient approximation algorithms for point-to-point channel coding results in the absence of non-signaling assisted capacity advantage.
