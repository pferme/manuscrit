% vim: spl=fr
\chapter*{Abstract}
\addcontentsline{toc}{chapter}{Abstract}
%\addcontentsline{tof}{chapter}{Résumé}
Finding efficient codes is crucial for achieving reliable communication over noisy channels. While error-correcting codes for i.i.d. channels are well understood, the task of approximating the best code for a general channel, i.e. that maximizes the success probability of the encoding and decoding process, is much less developed. This thesis addresses this coding problem in several scenarios.\vspace{2mm}\\
For point-to-point channels, optimal approximation algorithms for coding and list-decoding are known. We study a generalization of the latter with a variable list size, compensated by an increasing probability of failure. More generally, we study the coverage problem which entails maximizing $\sum_{a \in [n]} w_a \varphi(\abs{\set{i \in S : a \in T_i}})$ over subsets $S \subseteq [m]$ of cardinality $k$, for a general nondecreasing concave function $\varphi$. We provide an approximation algorithm for that problem of ratio $\alpha_{\varphi} := \min_{x \in \mathbb{N}^*} \frac{\mathbb{E}[\varphi(\Poi(x))]}{\varphi(\mathbb{E}[\Poi(x)])}$, which cannot be improved for sublinear $\varphi$ if $\textrm{P}\not=\textrm{NP}$.\vspace{2mm}\\
For multiple-access channels, we show that the coding problem cannot be approximated within any constant ratio under a complexity hypothesis on random $k$-SAT formulas. Generalizing and abstracting quantum entanglement, non-signaling correlations can be used to enhance communication. We show that optimal non-signaling assisted codes for multiple-access channels can be found in polynomial time in the number of copies of the channel. Applied to the binary adder channel, a non-signaling advantage on its capacity region is established. We provide a general single-letter outer bound on the non-signaling capacity region. When non-signaling assistance is not shared between encoders, we show that the capacity region is not changed.\vspace{2mm}\\
For broadcast channels, when restricted to deterministic channels, we provide a $(1-e^{-1})^2$-approximation algorithm for the unassisted coding problem, and we show that their capacity region is not changed with non-signaling assistance. In the value query model, we show that we cannot achieve a better approximation ratio than $\Omega\left(\frac{1}{\sqrt{m}}\right)$ for the general broadcast channel coding problem, with $m$ the output size of the channel.

\clearpage 

\flushright
\bigskip
\hrule \bigskip \bigskip
{\Huge \textbf{\textsf{Résumé}}}
\flushleftright
%\chapter*{Abstract}
\phantomsection
\addcontentsline{tof}{chapter}{Résumé}
%\addcontentsline{tof}{chapter}{Résumé en anglais}
\vspace{1.5cm}

\begin{otherlanguage}{french}
  La recherche de codes efficaces est essentielle pour obtenir des communications fiables sur des canaux bruités. Alors que les codes correcteurs d'erreurs pour les canaux i.i.d. sont bien compris, le problème d'approximer le meilleur code pour un canal générique, c'est-à-dire qui maximise la probabilité de succès de la communication sur les canaux bruités, a été beaucoup moins étudié. Cette thèse aborde ce problème de codage dans plusieurs scénarios.
  
  Pour les canaux point-à-point, des algorithmes d'approximation optimaux pour le problème du codage ainsi que le problème du décodage de liste sont connus. Nous étudions une généralisation de ce dernier avec une taille de liste variable, compensée par une probabilité d'échec croissante. Plus généralement, nous étudions le problème de couverture qui consiste à maximiser $\sum_{a \in [n]} w_a \varphi(\abs{\set{i \in S : a \in T_i}})$ sur les sous-ensembles $S \subseteq [m]$ de cardinal $k$, pour $\varphi$ une fonction croissante concave quelconque. Nous proposons un algorithme d'approximation pour ce problème de ratio $\alpha_{\varphi} := \min_{x \in \mathbb{N}^*} \frac{\mathbb{E}[\varphi(\Poi(x))]}{\varphi(\mathbb{E}[\Poi(x)])}$, qui ne peut être amélioré pour $\varphi$ sous-linéaire si $\textrm{P}\not=\textrm{NP}$.
  
  Pour les canaux à accès multiple, nous montrons que le problème du codage ne peut être approximé avec un ratio constant sous une hypothèse de complexité sur des formules $k$-SAT aléatoires. En généralisant et en abstrayant l'intrication quantique, les corrélations non-signalantes peuvent être utilisées pour améliorer la communication. Nous montrons que les codes optimaux avec assistance non-signalante pour les canaux à accès multiples peuvent être trouvés en temps polynomial en le nombre de copies du canal. Appliqué au canal additionneur binaire, l'utilisation de corrélations non-signalantes étend sa zone de capacité. Nous fournissons une borne supérieure sur la zone de capacité avec assistance non-signalante. Lorsque l'assistance non-signalante n'est pas partagée entre les encodeurs, nous montrons que la zone de capacité n'est pas modifiée.
  
  Pour les canaux de diffusion, lorsque l'on se limite aux canaux déterministes, nous fournissons une $(1-e^{-1})^2$-approximation pour le problème du codage sans assistance, et nous montrons que leur zone de capacité n'est pas modifiée par l'assistance non-signalante. Dans le modèle d'accès par valeur, nous montrons que nous ne pouvons pas obtenir un meilleur ratio d'approximation que $\Omega\left(\frac{1}{\sqrt{m}}\right)$ pour le problème global du codage des canaux de diffusion, où $m$ est la taille de sortie du canal.
\end{otherlanguage}
