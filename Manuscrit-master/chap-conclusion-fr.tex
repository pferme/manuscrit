Dans cette thèse, nous avons étudié plusieurs problèmes de codage de canal, avec et sans assistance non-signalante, du point de vue de l'approximation algorithmique et des zones de capacité.

Dans le Chapitre \ref{chap:MaxCoverage}, nous avons introduit une généralisation du problème de couvermture multiple maximale de paramètre $\ell$ où l'on maximise $\sum_{a \in [n]} w_a\varphi(\abs{\set{i \in S : a \in T_i}})$ sur les sous-ensembles $\subseteq [m]$ de cardinal $k$. Nous avons montré que lorsque $\varphi$ est normalisée, croissante et concave, nous pouvons obtenir une garantie d'approximation donnée par le \emph{rapport de concavité de Poisson} $\alpha_{\varphi} := \min_{x \ dans \mathbb{N}} \frac{\mathbb{E}[\varphi(\Poi(x))]}{\varphi(\mathbb{E}[\Poi(x)])}$ et nous avons montré qu'il est optimal pour les fonctions sous-linéaires $\varphi$ si $\textrm{P}\not=\textrm{NP}$. Appliqué au codage de canal, et plus spécifiquement au décodage de liste de paramètre $\varphi$, où la restriction de longueur sur la liste de suppositions du décodeur dans le décodage de liste de paramètre $\ell$ est remplacée par une probabilité $\frac{\varphi(\ell)}{\ell}$ de décoder correctement une liste de suppositions de taille variable $\ell$, nous avons obtenu une stricte garantie d'approximation $\alpha_{\varphi}$ pour la classe de canaux $W$ de la forme $W(y|x) = \frac{1}{t}$ pour $y \in T_x$ avec $\abs{T_x}=t$ et $W(y|x) = 0$ ailleurs. Une question ouverte naturelle est de savoir si la garantie de \textrm{NP}-difficulté peut être étendue pour $\varphi(n) \not= o(n)$. Un autre problème ouvert consiste à étendre l'algorithme d'approximation $\alpha_{\varphi}$ pour le décodage de listes de paramètre $\varphi$ sur tous les canaux.

Dans le Chapitre \ref{chap:MAC}, nous avons montré que le problème du codage des canaux à accès multiple ne peut être approximé avec un ratio constant sous certaines hypothèses de complexité sur les clauses aléatoires. Nous avons montré que les codes optimaux non-signalants pour les canaux à accès multiple peuvent être trouvés en temps polynomial en le nombre de copies du canal. Appliqué au canal additionneur binaire, un avantage non-signalant sur sa zone de capacité a été établi. Nous avons fourni une borne supérieure générale à une seule lettre pour la zone de capacité non-signalante. Lorsque l'assistance non-signalante n'est pas partagée entre les encodeurs, nous avons montré que la zone de capacité n'est pas modifiée. La question de savoir si l'intrication quantique peut augmenter la capacité du canal additionneur binaire reste ouverte. En outre, l'établissement d'une formule à une seule lettre pour la capacité avec assistance non-signalante des canaux à accès multiple reste la principale question ouverte de ce chapitre. Cette dernière pourrait être obtenue en montrant que la borne supérieure à une seule lettre fournie pour la zone de capacité non-signalante peut en fait être atteinte.

Dans le Chapitre \ref{chap:Broadcast}, nous avons montré que l'assistance non-signalante partagée uniquement entre les décodeurs ne modifie pas la zone de capacité des canaux de diffusion. En se limitant aux canaux déterministes, nous avons fourni un algorithme d'approximation de ratio $(1-e^{-1})^2$ pour le problème du codage sans assistance, et nous avons montré que leur zone de capacité n'est pas modifiée par l'assistance non-signalante. Dans le modèle d'accès par valeur, nous avons montré que nous ne pouvons pas obtenir un meilleur ratio d'approximation que $\Omega\left(\frac{1}{\sqrt{m}}\right)$ pour le problème de codage du canal de diffusion général, avec $m$ une taille de sortie du canal. Nos résultats suggèrent que l'assistance non-signalante pourrait améliorer la zone de capacité des canaux de diffusion généraux, ce qui reste une question ouverte majeure. Un résultat intermédiaire consisterait à montrer qu'il est \textrm{NP}-ifficile d'approximer le problème de codage des canaux de diffusion pour n'importe quel ratio constant, ce qui renforcerait notre résultat de difficulté sans s'appuyer sur le modèle d'accès par valeur.

Tout au long de cette thèse, nous avons montré que les différents liens entre les algorithmes d'approximation pour le codage de canal et les zones de capacité avec assistance non-signalantes sont fructueux dans les deux sens. D'une part, l'existence d'algorithmes d'approximation à rapport constant pour le codage de canal se traduit souvent par l'absence d'avantage non-signalant pour les zones de capacité correspondantes. D'autre part, la difficulté d'approximation pour un ratio constant quelconque pour le problème du codage de canal se traduit souvent par l'existence d'un avantage non-signalant dans les zones de capacité correspondantes. Bien qu'il ne s'agisse pas d'une équivalence formelle, ce lien étroit entre ces domaines apparemment sans rapport peut être la clé pour résoudre des problèmes non résolus dans les deux domaines. En particulier, l'étape franchie dans la compréhension de l'influence des corrélations non-signalantes sur les canaux de diffusion, résolue à présent pour les canaux déterministes, va dans cette direction et pourrait être étendue à d'autres classes de canaux. Nous pensons que cette connexion mérite plus d'attention, pour tout problème de codage de canal et même plus généralement pour tout jeu à plusieurs joueurs tel que le jeu CHSH, car les approches radicalement différentes des deux domaines seront très probablement profitables pour une compréhension plus approfondie de leurs questions ouvertes respectives.
