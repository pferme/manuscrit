%Need for communication with increasing number of connected devices eg.

Limits and capacity region established by Shannon~\cite{Shannon48}, capacity, fundamental limits. Howevere only theoritical.

How to effectively achieve these rates: one common approach is by finding error-correcting codes (BIBLIO TODO), multiple copies of a channel.

Another POV is the algo task of finding the best code for one copy of the channel, maximizing the success sending a fixed number of messages~\cite{BF18}. Task NP-hard, best approx in poly time in $1-e^{-1}$ and cannot do better. Task studied also for list-decoding~\cite{Elias57,Wozencraft58} in~\cite{BFGG20} and more generally for a larger class of combinatorial optimization problems.

Another POV on sending information that arise is the use of quantum ressources. Small and concise intro on quantum machanics. QE shared helps sending info while not being able to send info directly (BIBLIO TODO).

A more general and abstract notion on ressource is non-signaling assistance. Explain, theoritical but useful, example of the CHSH game where it is stronger than quantum.

Back to~\cite{BF18}: finding the best code relies in fact in finding a non-signaling code and transforming back into a classical code. Easy because NS is linear program. Also implies that the capacity regions are the same.

Network channels intro, modelize complex situations where multiple inputs outputs. Update on what is known on Q here. (take common part of intros on MACs and Broadcast here)

In this thesis, address the problem of finding efficient codes for several communication scenarios. First, address algo question motivated by a generalization of List-decoding and \cite{BFGG20}, as well as the more abstract algo problem (RK here not so interesting to look at NS as same capacity ? Discuss with Omar). Then problem studied for network scenarios of MACs, Broadcast channels, as well as their link to their NS capacities.

Section results and summary of results.


\newpage

With the growing number of connected devices and the explosion in the amount of exchanged data, the need for efficient and reliable communication has never been as critical as in today's world. Noise, coming from physical imperfection, signal interference, or even \emph{lepidopterans}, is the major hurdle to overcome.

The Theory of Information established by Shannon in his seminal work~\cite{Shannon48} provides clear and definite answers to the asymptotic amount of data that can be transmitted through noisy point-to-point channels.
